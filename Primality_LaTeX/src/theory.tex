\section{Theory}
There exist several algorithms to check whether a given integer, $n$, is a prime number.
The easiest among them all is the ``trial division'' (TD) algorithm, described in \secref{sec:TD_theory}.
It is a deterministic algorithm with a $O(\sqrt{n})$ complexity.
Several different algorithms have been devised, both deterministic and random, to beat this complexity.
All randomized algorithms below are based on ``Fermat's Theorem'' for primality \cite{book:RA}, which says that ``for every prime number, $n$,
\begin{equation}
	a^{n-1} \equiv 1 \mathrm{(mod}\, n\, \mathrm{)},
\end{equation}
$\forall a \in Z_n^*$''.
Sadly, there does as well exist an infinite set of composite numbers which satisfy this criterion.
They are the so-called ``Carmichael numbers''.
Each of the randomized algorithms below deals with these numbers in their own way.

\subsection{Trial Division} \label{sec:TD_theory}
In the trial division (TD) algorithm, we start with the very definition of a prime number: it is only dividable by 1 and itself.
To test this statement, we divide by every integer up to $\sqrt{n}$.
If any of those divisions result in an integer, the number is not a prime.
Otherwise, it must necessarily be a prime, from the very definition.
TD is a deterministic algorithm with $O(\sqrt{n})$ complexity.

\subsection{Wheel-Sieve} \label{sec:WS_theory}
The Wheel-Sieve (WS) algorithm is an optimized version of TD.
The algorithm takes the first $k$ prime numbers (hard-coded, or with a cheap primality algorithm) $p_1=2, p_2, \ldots, p_k$ as initial prime numbers.
First the algorithm uses trial division for these $k$ numbers to see whether $n$ is dividable by any of these numbers.

Since we only take the first $k$ prime numbers, which may be very far off from $n$, further steps are required.
Let, $m = \prod p_i$,
then the algorithm will test divisibility of $n$ by all numbers $l$, such that $\forall p_i: l \neq p_i (\mod m)$.
Unfortunately, this algorithm gives only a linear speedup in comparison with trial division.

\subsection{Solovay-Strassen} \label{sec:SS_theory}
The Solovoy-Strassen (SS) algorithm is a randomized primality test based on Fermat's Theorem.
The algorithm will never error for prime $n$,
but it has a probability of at most $(\frac{1}{2})^k$ of incorrectly identifying a composite number as prime when the algorithm is repeated $k$ times.
Each trial of the algorithm has a runtime of $O((\log n)^3)$ as this is the time needed for modular exponentiation.

\subsection{Miller-Rabin} \label{sec:MR_theory}
The Miller-Rabin (MR) primality test is also based on Fermat's Theorem and will never error for prime $n$ either.
On composite numbers it will error with a probability smaller than $\frac{1}{4}$. \cite{MR}
Again each trial of the algorithm has a runtime of $O((\log n)^3)$ as this is the time needed for modular exponentiation.

\subsection{Angrawal-Kayal-Saxena} \label{sec:AKS_theory}
Angrawal-Kayal-Saxena (AKS) is a polynomial deterministic primality test, published in 2002.
They were the first to show that primality is in $P$.
The authors state that the runtime of this algorithm is at most $O((\log n)^{\frac{21}{2}})$ \cite{AKS}.


